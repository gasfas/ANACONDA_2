\section{Metadata structure}
The metadata contains all information needed to correct, convert, fit and visualize the data.
\\
The metadata structure is divided into different categories:
\begin{itemize}
\item[\emph{sample}], e.g. atomic mass, expected fragment masses, constituent masses.
\item[\emph{photon beam}], e.g. the photon energy, intensity, duration, etc
\item[\emph{spectrometer}],  e.g. the name, voltages and relevant dimensions of the used spectrometer are listed here.
\item[\emph{detectors}], e.g. the names and properties of the detectors are stored in here.
\item[\emph{correct}] The parameters needed to execute corrections onto the raw data, before conversion. For example, translation in X and Y to move the centre of detection into the origin of the coordinate system.
\item[\emph{calibrate}] The information needed to perform the calibrations. Note that these are not the actual calibration factors, they are stored in the 'convert' field.
\item[\emph{fit}] The fitting parameters.
\item[\emph{convert}] The conversion factors (sorted in terms of detectors), such as mass to charge conversion.
\item[\emph{plot}] The user-preferred plotstyle.
\end{itemize}

Most fields in the metadata are obvious to understand. We elaborate on a few that might be cause of confusion.

\paragraph{sample}
 We elaborate on the definition difference between 'constituents' and 'fragments' here:
\begin{itemize}
\item Constituents: The building block of which the sample consists. Example: a water-ammonia mixed cluster has consituents water and ammonia
\item Fragments: The expected fragments from the sample. Example: a water-ammonia mixed cluster has the expected fragments of hydrogenated water-ammonia mixed clusters.
\end{itemize}


\subsection {storage}
The metadata is stored as a struct, with the above categories as their fieldnames. 

The metadata, or data settings, belong to a separate datafile. They are stored in a separate file with the same base name as the main datafile (\emph{'filename.mat'}), but with the addition \emph{'md\_.m'} before the filename, so \emph{'md\_filename.m'}. This is done to make it easy to manually copy metadata files to different datafiles. For example: reading the datafile \emph{H2O\_003.mat} will read its metadata from the file \emph{md\_H2O003.m}. The metadata is stored in a plain-text m-file, which makes it possible to read and change parameters from outside MATLAB.

\lstset{language=MATLAB}
\begin{lstlisting}
exp1_md.sample
exp1_md.photon
exp1_md.spec
exp1_md.corr
exp1_md.calib
exp1_md.fit
exp1_md.conv
exp1_md.plot
\end{lstlisting}
